\documentclass[draft,12pt,headsepline,footsepline,paper=letter]{scrreprt}
\pagestyle{headings}

\usepackage[utf8]{inputenc}
\usepackage[T1]{fontenc}
\def\spanishoptions{es-noquoting,es-nolists,mexico-com}
\usepackage[spanish]{babel}

\usepackage{makeidx}
\makeindex

\usepackage[nonumberlist]{glossaries}
\makeglossaries

\usepackage[final]{graphicx}
\DeclareGraphicsExtensions{.pdf,.png,.jpg}
\graphicspath{{media/}}

\iftrue % outline for images
\usepackage{scrhack}
\usepackage{float}
\floatstyle{boxed}
\restylefloat{figure}
\fi

\usepackage{natbib}
\usepackage{amsmath,amssymb, bm}
\usepackage{enumerate}
\usepackage{ragged2e}

\usepackage{setspace}
\onehalfspacing
\frenchspacing
\recalctypearea

\usepackage{pdfsync}

\begin{document}

\title{Protocolo de Investigación: Calendarización de horarios educacionales}
\author{Héctor Arciga}
\date{\today}

\maketitle

\begin{spacing}{1}
\tableofcontents
\glsaddall 
\printglossaries
\listoffigures
\listoftables
\end{spacing}

% Contenido

\chapter{Introducción}

% Capítulo: Introducción

\section{Preguntas de investigaci\'on}

Preguntas estrat\'egicas

- ¿Cuál es el impacto de los horarios escolares en una institución educacional?
- ¿Qué actores están involucrados en el diseño de un horario escolar?
- ¿Cuáles son los limites de un horario educacional?

Preguntas de investigación

- ¿Cómo se define un horario escolar?
- ¿Qué elementos participan en la elaboración de un horario escolar?
- ¿Qué beneficios conlleva el plantamiento formal de un problema de calendarización de un horario escolar?

\chapter{Objetivos de la investigación}

\index{gestor escolar}\index{métricas de desempeño}\index{formulación matemática}
Esta tesis investiga la problemática de la elaboración de horarios en instituciones docentes y el impacto potencial de su formulación en distintas métricas de desempeño educacional relevantes para el gestor escolar. 
El fin primordial del trabajo es investigar de qué manera puede el gestor escolar hacer uso de estas técnicas, para la consecución de sus estrategias.
Para poder cumplir con esta finalidad, los siguientes objetivos son considerados:
\begin{enumerate}[1]
\item La investigación de las distintas problemáticas en la planificación de horarios en instituciones docentes.
\item Las distintas formulaciones matemáticas de los problemas de planificación de horarios.
\item Los algoritmos de solución disponibles para este tipo de modelos matemáticos.
\item Los antecedentes históricos de técnicas de optimación aplicados a partir del surgimiento de los ordenadores.
\item La recopilación de las métricas de desempeño educacional relevantes para los gestores escolares.
\item Una investigación de las limitantes más comunes utilizadas en las formulaciones de los problemas de planificación de horarios.
\end{enumerate}

\bibliography{/Users/harciga/Dropbox/bibliographies/reviewed}
\end{document}
