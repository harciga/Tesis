\documentclass[draft,12pt,headsepline,footsepline,paper=letter]{scrreprt}
\pagestyle{headings}

\usepackage[utf8]{inputenc}
\usepackage[T1]{fontenc}
\def\spanishoptions{es-noquoting,es-nolists,mexico-com}
\usepackage[spanish]{babel}

\usepackage{makeidx}
\makeindex

\usepackage[nonumberlist]{glossaries}
\makeglossaries

\usepackage[final]{graphicx}
\DeclareGraphicsExtensions{.pdf,.png,.jpg}
\graphicspath{{media/}}

\iftrue % outline for images
\usepackage{scrhack}
\usepackage{float}
\floatstyle{boxed}
\restylefloat{figure}
\fi

\usepackage{natbib}
\usepackage{amsmath,amssymb, bm}
\usepackage{enumerate}
\usepackage{ragged2e}

\usepackage{setspace}
\onehalfspacing
\frenchspacing
\recalctypearea

\usepackage{pdfsync}

\begin{document}

\title{Protocolo de Investigación: Calendarización de horarios educacionales}
\author{Héctor Arciga}
\date{\today}

\maketitle

● Introduction
● The objectivesand purpose of the project (what?):
– A brief overall description of the project’scontext.
– The strategic question that guidesthe project.
– The objectivesof the project.
● The justification for the project (why?).
● The research questions(what? – again, but in more detail):
– Identify and discussthe research questionsthat you will answer in the
project.
– If you are taking a realist research approach you might frame your
research question asa hypothesis. A hypothesisisa speculation about
an association between two or more variables. It isimportant that the
hypothesiscan be tested by research to see whether it can be
disproved. 
● An overview of the appropriate literature:
– Mapping the main writersin the field and their arguments.
– Definition of key conceptsand outlining of conceptual frameworkif
necessary and possible. (Thisisthe how– conceptually?quadrant of
the Watson Box.)
● Research design:
– What methodological approach are you going to adopt?
–Research methods, samples, methodsfor analysing research material.
● Practical and ethical issues:
–Doesthe research raise any ethical concernsthat need to be resolved?
– Are there any potential problemsof research access?
– Are there any resource issues such asaccessto specialist databasesor
particular research software?
– Are there issuesof commercial confidentially or intellectual property
rights?
● A plan or timetable:
– Consider drafting a Gantt chart (as shown in Exhibit 1.11) that plots
against a timeline when the major elementsof the project will be done.

\section{Introducción}


\section{Objetivos de la investigación}
– A brief overall description of the project’scontext.
– The strategic question that guidesthe project.
– The objectivesof the project.
\index{gestor escolar}\index{métricas de desempeño}\index{formulación matemática}
Esta tesis investiga la problemática de la elaboración de horarios en instituciones docentes y el impacto potencial de su formulación en distintas métricas de desempeño educacional relevantes para el gestor escolar. 
El fin primordial del trabajo es investigar de qué manera puede el gestor escolar hacer uso de estas técnicas, para la consecución de sus estrategias.
Para poder cumplir con esta finalidad, los siguientes objetivos son considerados:
\begin{enumerate}[1]
\item La investigación de las distintas problemáticas en la planificación de horarios en instituciones docentes.
\item Las distintas formulaciones matemáticas de los problemas de planificación de horarios.
\item Los algoritmos de solución disponibles para este tipo de modelos matemáticos.
\item Los antecedentes históricos de técnicas de optimación aplicados a partir del surgimiento de los ordenadores.
\item La recopilación de las métricas de desempeño educacional relevantes para los gestores escolares.
\item Una investigación de las limitantes más comunes utilizadas en las formulaciones de los problemas de planificación de horarios.
\end{enumerate}
\section{Justificación}

Preguntas estratégicas

- ¿De qué manera se puede elaborar un horario escolar para incidir en el desempeño de una institución educacional de una manera en particular?

\section{Preguntas de investigación}

- ¿Qué es un horario escolar?
- ¿Qué elementos componen un horario escolar?
- ¿Cómo se elabora un horario escolar? 
- ¿Qué elementos participan en la elaboración de un horario escolar?
- ¿Cuáles son las métricas de desempeño relevantes para una institución educacional?
- ¿De qué manera el horario escolar puede incidir en tales métricas de desempeño?

\section{Marco teórico}
\section{Diseño de la investigación}
\section{Cuestiones prácticas y éticas}
\section{Calendario de la investigación}
\bibliography{/Users/harciga/Dropbox/bibliographies/reviewed}
\end{document}
